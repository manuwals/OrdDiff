\section{Kapitel 3}

\subsection{Inhomogen 2. Ordnung}
\begin{tabbing}
1. Homogenes System lösen \= $a \cdot y''(t) + b \cdot y'(t) + c \cdot y(t) = 0$\\
2. Inhomogener Ansatz \> $a \cdot y''(t) + b \cdot y'(t) + c \cdot y(t) = g(t)$\\
\>\begin{tabular}{|l|l|}
	\hline
	\textbf{Inhomogenität}              & \textbf{Ansatz} \\ \hline
	$g(t)=a_0\,t^n + ... + a_n$        & $y(t) = t^s(A_0\,t^n + ... + A_n)$ \\ \hline
	$g(t)=(a_0\,t^n + ... + a_n)e^{\alpha \, t}$  & $y(t) = t^s(A_0\,t^n + ... + A_n)e^{\alpha \, t}$ \\ \hline
	$g(t)=(a_0\,t^n + ... + a_n)e^{\alpha \, t}sin(\beta \, t)$
		& $y(t) = t^s(A_0\,t^n + ... + A_n)e^{\alpha \, t}sin(\beta \, t) + $ \\ 
	oder $(a_0\,t^n + ... + a_n)e^{\alpha \, t}cos(\beta \, t)$ 
		&$t^s(B_0\,t^n + ... + B_n)e^{\alpha \, t}cos(\beta \, t)$ \\ \hline
\end{tabular}\\
\> $s = [0,1,2]$, so dass kein Summand Lösung der homogenen Gleichung.\\
3. Zusammensetzen \> $y(t)=y_h(t)+y_p(t)$
\end{tabbing}

\subsection{Schwingungen}

\begin{tabular}{|l|l|}
	\hline
	\textbf{Terminologie}              & \textbf{Gleichung} \\ \hline
	freie ungedämpfte Schwingung       & $m \cdot y''(t) + k \cdot y(t) = 0$ \\ \hline
	freie gedämpfte Schwingung         & $m \cdot y''(t) + \gamma\cdot y'(t) + k \cdot y(t) = 0$ \\ \hline
	erzwungene ungedämpfte Schwingung  & $m \cdot y''(t) + k \cdot y(t) = F(t)$ \\ \hline
	erzwungene gedämpfte Schwingung    & $m \cdot y''(t) + \gamma\cdot y'(t) + k \cdot y(t) = F(t)$ \\ \hline
\end{tabular}\\

\textbf{Matrixschreibweise}\\
$ y''(t) + p(t)\cdot y'(t) + q(t) \cdot y(t) = g(t) \rightarrow x' = A \cdot x + b \quad 
A(t)=\begin{bmatrix}
 0 & 1 \\
 -q(t) & -p(t)\\
\end{bmatrix} \quad
b(t)=\begin{bmatrix}
 0 \\
 g(t)\\
\end{bmatrix}$

\subsubsection{Freie Schwingung}
$m \cdot y''(t) + \gamma\cdot y'(t) + k \cdot y(t) = 0$\\
\begin{tabbing}
ohne Dämpfung ($\gamma = 0$): \= $y(t) = A \cdot cos(w_0 \cdot t) + b\cdot sin(w_0 \cdot t) \quad w_0=\sqrt{\dfrac{k}{m}} \quad  A=y_0 \quad B=\dfrac{y_0'}{w_0}$\\
\>$y(t) =R\cdot cos(w_0 \cdot t - \varphi) \quad R=\sqrt{A^2+B^2} \quad \varphi=atan2(B,A)$\\
\end{tabbing}

\begin{tabbing}
mit Dämpfung ($\gamma \neq 0$): \= $Z(\lambda)=m\lambda^2 + \gamma \lambda + k \rightarrow \lambda_{1,2}=\dfrac{-\gamma \pm \sqrt{\gamma^2 - 4 \, m \, k}}{2 \, m} \rightarrow D=\gamma^2 - 4 \, m \, k$\\
\> Fall 1: $D>0 \quad y(t)=Ae^{\lambda_1 \, t} + Be^{\lambda_2 \, t}$\\
\> Fall 2: $D=0 \quad y(t)=(A + B \,t)e^{\lambda \, t}$\\
\> Fall 3: $D<0 \quad y(t)=Ae^{\mu \, t}cos(\nu \, t) + Be^{\mu \, t}sin(\nu \, t) \quad \mu=-\dfrac{\gamma}{2\,m} \quad \nu = \sqrt{\dfrac{k}{m}-\dfrac{\gamma^2}{4 \, m^2}}$
\end{tabbing}

$\operatorname{atan2}(B, A) = \begin{cases}
\arctan\left(\frac B A\right) & \qquad A > 0 \\
\arctan\left(\frac B A\right) + \pi& \qquad B \ge 0 , A < 0 \\
\arctan\left(\frac B A\right) + 2\pi& \qquad B < 0 , A < 0 \\
+\dfrac{\pi}{2} & \qquad B > 0 , A = 0 \\
+\dfrac{3\pi}{2} & \qquad B < 0 , A = 0 \\
\text{undefined} & \qquad B = 0, A = 0
\end{cases}$

\subsubsection{Erzwungene Schwingung}


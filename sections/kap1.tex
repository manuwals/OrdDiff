\section{Kappitel 1}

\subsection{Symbolisch}

\subsubsection{Allgemein 1. Ordnung}
$\dfrac{d}{dt}y(t)=f(t,y(t))$

\subsubsection{Linear}
\begin{tabular}{p{6cm}p{1.8cm}p{10.5cm}}
\textbf{Form:} $\dfrac{d}{dt}y(t) +p(t)y(t)=g(t)$ &
\textbf{Vorgehen:}              &
1. $\mu(t)=e^{\int p(t)dt}$ \\ &&
2. $y(t)=\dfrac{1}{\mu(t)}\left(\int{\mu(t)g(t) dt+C}\right)$ \\ &&
\end{tabular}

\subsubsection{Separierbar}
\begin{tabular}{p{6cm}p{1.8cm}p{10.5cm}}
\textbf{Form:} $M(x)+N(y(x))(\dfrac{d}{dx}y(x))=0$ &
\textbf{Vorgehen:}              &
1. $H_1=\int M(x)dx \qquad H_2=\int N(y)dy$ \\ &&
2. $H_1(x) + H_2(y(x)) = C$ \\ &&
\end{tabular}

\subsubsection{Exakt}
\begin{tabular}{p{6cm}p{1.8cm}p{10.5cm}}
\textbf{Form:} $M(x,y) + N(x,y)(\dfrac{d}{dx}y(x))=0$ &
\textbf{Vorgehen:}              &
1. $ M(x,y) = \dfrac{d}{dx}\Psi(x,y) \qquad N(x,y) = \dfrac{d}{dy}\Psi(x,y) $ \\
wobei $\dfrac{d}{dy}M(x,y) = \dfrac{d}{dx}N(x,y)$ &&
2. Potential $\Psi(x,y) = Q(x,y) + h(y) \qquad Q(x,y) = \int M(x,y) dx$ \\ &&
3. $\dfrac{d}{dy}h(y) = N(x,y) - \dfrac{d}{dy}Q(x,y) \rightarrow h(y)=\int dy$  \\ &&
4. Lösung: $\Psi(x,y) = Q(x,y) + h(y) = c$\\ &&
\end{tabular}

\subsection{Nummerisch}
\subsubsection{Fehler}

\begin{tabbing}
Tatzächlicher Wert  \= $\phi(t_n)$ \\
Globaler Fehler \> $E_n = \phi(t_n) - y_n$ \\
Lokaler Fehler \> $e_{n+1} = \phi(t_n + h) - y_{n+1}$ \\
Rundungsfehler \> $R_n = y_n - Y_n$ wobei $Y_n$ gerundet
\end{tabbing}

\subsubsection{Euler}
Polygonzug mit Steigung $y'(t)$\\
$y'(t)=f(t,y(t)) \qquad y(t_0)=y_0 \qquad$ Zeitschritt $h$\\
$t_i = t_{i-1} + h \qquad y_i=y_{i-1} + h \cdot f(t_{i-1},y_{i-1})$\\

$|\phi''(t)|<M \qquad |e_n| < \dfrac{M*h^2}{2} \qquad
h_n < \sqrt{\dfrac{2\epsilon}{M}} \qquad E_n \approx h$
\subsubsection{Heun}
Trapez Approximation\\
$y'(t)=f(t,y(t)) \qquad y(t_0)=y_0 \qquad$ Zeitschritt $h$\\
Idee: $y_i=y_{i-1} + h \cdot \dfrac{f(t_{i-1},y_{i-1}) + f(t_{i},y_{i})}{2}$\\
$t_i = t_{i-1} + h \qquad y_{i-1} + h \cdot \dfrac{f(t_{i-1},y_{i-1}) + f(t_{i},y_{i-1} + h \cdot f(t_{i-1},y_{i-1}))}{2}$\\
$e_n \approx h^3 \qquad E_n \approx h^2$
\subsubsection{Runge-Kutta}
$y_{i+1}=y_i + h \cdot \dfrac{k_{i,1} + 2\:k_{i,2} + 2\:k_{i,3} + k_{i,4} }{6}$\\
$k_{i,1} = f(t_{i},y_{i}) \qquad 
k_{i,2} = f(t_{i} + \dfrac{h}{2},y_{i} + \dfrac{h}{2}k_{i,1}) \qquad 
k_{i,3} = f(t_{i} + \dfrac{h}{2},y_{i} + \dfrac{h}{2}k_{i,2}) \qquad 
k_{i,3} = f(t_{i} + h,y_{i} + h \: k_{i,2}) \qquad$\\
$e_n \approx h^5 \qquad E_n \approx h^4$
\subsection{Bifurkationen}

\section{Kapitel 2}

\subsection{Begriffe}
\subsubsection{Vektorfeld und Trajektorie}
Hat man ein autonomes System folgender Art:
\begin{equation*}
\frac{d}{dt}x(t) = f(x(t))
\end{equation*}
Dann ist $x(t) = [x_1(t), x_2(t)]$ eine Lösung sowie ein Vektor, welcher von der Zeit t abhängt. Die Funktion $f$ ordnet dem Vektor $x$ den Vektor $f(x) = [f_1(x_1,x_2),f_2(x_1,x_2)]$ (= partielle Ableitung in $x_1$ und $x_2$) zu. 
Die Darstellung dieser Vektoren $f_1$ und $f_2$ in der $x_1,x_2$ Ebene (= \textbf{Phasenraum}) ist das assoziierte Vektorfeld. 
Die entstehenden Kurven in der Phasenebene werden \textbf{Trajektorien} genannt. 
Der \textbf{Zustandsraum} $x_1,x_2,t$ beinhaltet alle Lösungskurven eines autonomen Systems.
Die räumlichen Lösungskurven im Zustandsraum sind translationsinvariant, d.h eine Verschiebung in t-Richtung liefert wieder eine Lösungskurve des Systems. Bei festen Anfangswerten $x_0, y_0$ unterscheiden sich Lösungskurven nur durch eine Verschiebung in t-Richtung. Die Normalprojektion einer solchen Schar paralleler Lösungen in den Phasenraum ist eine Trajektorie. 
Im Phasenraum interessieren vor allem die singulären Punkte. Die Nullstellen des Feldes, sind die Normalprojektionen der stationären Lösungen, d.h. den Gleichgewichtslösungen, in den Phasenraum. 
Die folgenden Bilder zeigen das Vektorfeld sowie verschiedene Trajektorien (sprich verschiedene Anfangsbedingungen) im Phasenportrait.\\  
Der Zustandsraum wäre die Darstellung des Phasenraumes mit einer dritten Achse (ins Blatt hinein), welche den zeitlichen Verlauf darstellt.\\
\subsubsection{Richtungsfeld und Lösung - Begriff der Lösung}
Die Lösung der Differentialgleichung $\frac{d}{dt}x(t) = f(t,x(t))$ ist eine parametrisierte Kurve x(t). Die Ableitung nach t ist demnach der Tangentenvektor an diese Kurve. Die Diffgleichung enthält demnach die Information über die Tangentenvektoren an ihre Lösungskurven. 
Das bedeutet, auch ohne die Lösung können wir die rechte Seite der Gleichung - d.h. die Tangentenvektoren - zeichnen. Das Lösen der Gleichung ist schlussendlich nur das Einpassen von Kurven in dieses Vektorfeld. Solche Kurven sind die Trajektorien der Gleichung. Die Trajektorie ist jedoch \textbf{nicht} die Lösung, da als Beispiel die Zeitkomponente fehlt. Hätte man (t,x(t)), hätte man die gesamte Lösung. 
\begin{minipage}[h]{0.35\textwidth} 
	\includegraphics[width=0.9\textwidth]{images/Vektorfeld.png}
\end{minipage}
\begin{minipage}[h]{0.35\textwidth}
	\includegraphics[width=1.0\textwidth]{images/Phasenportrait.png}
\end{minipage}

\subsection{Matrixdarstellung}
$x' = Ax + b$\\
Kritische Punkte: $x' = Ax + b = 0$ Gleichgewicht: $b = -Ax$ oder $x = -A^{-1}*b$\\


\subsection{Superpositionsprinzip}
Gegeben ist folgendes lineares homogenes Gleichungssystem:
\begin{equation*}
\frac{d}{dt}x(t) = A(t)x(t)
\end{equation*}
Das Superpositionsprinzip sagt, dass die Linearkombination der beiden Lösungen $x_1(t)$ und $x_2(t)$ auch wieder eine Lösung des Differentialgleichungssystems ist. 
\begin{equation*}
	x(t) = c_1 x_1(t) + c_2 x_2(t)
\end{equation*}
Wobei $c_1$ und $c_2$ beliebige Skalare sind. 
\subsection{Lineare Unabhängigkeit}
Die lineare Unabhängigkeit der beiden Lösungen $x_1(t)$ und $x_2(t)$ kann mit Hilfe der \textbf{Wronskideterminante} geprüft werden. 
\begin{equation*}
	W(t) = det[X(t)] =    
	\begin{vmatrix} 
	        x_{11}(t) & x_{12}(t)\\ 
	        x_{21}(t) & x_{22}(t)\\   
	\end{vmatrix}
\end{equation*}
Sind $x_1(t)$ und $x_2(t)$ \textbf{linear unabhängig}, dann gilt $W(t) \neq 0$ für alle $t \in I$. \\
Sind $x_1(t)$ und $x_2(t)$ \textbf{linear abhängig}, dann gilt $W(t) = 0$ für alle $t \in I$. \\
Jede Lösung $\Phi(t)$ kann als Linearkombination eines Fundamentalsystems zweier Lösungen $x_1(t)$ und $x_2(t)$ dargestellt werden. Diese Linearkombination wird als die allgemeine Lösung eines linearen homogenen Systems bezeichnet. 
\subsection{Fluss eines Vektorfeldes}
Gegeben ist ein autonomes lineares homogenes System mit einer nxn Matrix A:
\begin{equation*}
\frac{d}{dt}x(t) = A(t)x(t)
\end{equation*}
Die Matrix $A(t)$ besitzt verschiedene reelle Eigenwerte. Aus den Eigenwerten $\lambda_1$ und $\lambda_2$ und ihren zugehörigen Eigenvektoren $\nu_1$ und $\nu_2$ erhalten wir ein System von Fundamentallösungen: 
\begin{equation*}
x_1(t) = e^{\lambda_1t}\nu_1\\
\end{equation*}
\begin{equation*}
x_2(t) = e^{\lambda_2t}\nu_2
\end{equation*}
Die Fundamentallösungen $x_1(t)$ und $x_2(t)$ bilden die Spalten der Fundamentalmatrix $X(t)$:
\begin{equation*}
	X(t) =     
	\begin{vmatrix} 
	        x_{11}(t) & x_{12}(t)\\ 
	        x_{21}(t) & x_{22}(t)\\   
	\end{vmatrix}
\end{equation*}
Diese Fundamentalmatrix genügt folgender Differentialgleichung: 
\begin{equation*}
\frac{d}{dt}X(t) = A(t)X(t)
\end{equation*}
Fluss? Versteh ich nicht so ganz. 

Die Lösung x(t) des Anfangswertproblems $x(t_0) x_0$ kann mit Hilfe einer Fundamentalmatrix X(t) dargestellt werden. Die allgemeine Lösung der Diffgleichung $\frac{d}{dt}x(t) = A(t)x(t)$ kann als Superposition von Fundamentallösungen  geschrieben werden.\\
\begin{equation*}
x(t) = c_1 x_1(t) + ... + c_n x_n(t)
\end{equation*}
\begin{equation*}
x(t) = X(t)c
\end{equation*}
Mit $x_0 = X(t_0)$ und $det(X(t_0)) \neq 0$
\begin{equation*}
c = X(t_0)^{-1}x_0
\end{equation*}
In obige Gleichung eingesetzt, ergibt: 
\begin{equation*}
x(t) = X(t)X(t_0)^{-1}x_0
\end{equation*}
Wählt man nun die Fundamentallösungen $x_1(t) ... x_n(t)$ zu den Anfangsbedingungen $x_1(t_0)=e_1, ..., x_n(t_0)=e_n$, wobei $e_1...e_n$ die kanonischen Basisvektoren des Vektorraumes $R^n$ sind, erhalten wir die zugehörige Fundamentalmatri:
\begin{equation*}
\Phi(t_0) = E
\end{equation*}
Es folt aus dem Eindeutigkeitssatz, dass die Lösung x(t) des Anfangswertproblems $x(t_0) = x_0$ mit Hilfe dieser speziellen Fundamentalmatrix $\Phi(t)$, folgende Darstellung besitzt: 
\begin{equation*}
x(t) = \Phi(t)x_0
\end{equation*}
mit $\Phi(t) = X(t)X(t_0)^{-1}$. In Worten: Die Matrix A ordnet jedem Vektor x einen Vektor A(x) des Vektorfeldes zu. Der Fluss $\Phi(t)$ transformiert jeden Anfangszustand $x_0$ längs der druch $x_0$ verlaufenden Trajektorie in den Zustand x(t) zur Zeit t. 
Das Vektorfeld lässt sich auch aus dem Fluss zurück gwinnen, indem die Trajektorie $\phi(t) = \Phi(t)x$ nach er Zeit an der Stelle $t=0$ abgeleitet wird. 
\begin{equation*}
\frac{d}{dt}\Phi(t)x = A(x)
\end{equation*}
\subsection{Degenerierte Matrizen}
Ist A degeneriert, bedeutet dies, dass sie mindestens einen Eigenwert $\lambda$ mit einer geometrischen Vielfachheit besitzt, die kleiner als seine algebraische Vielfachheit ist. 
Das bedeutet, dass die linear unabhängigen Eigenvektoren der Matrix A den Raum $R^n$ nicht ausschöpft. 
\subsubsection{Theorem}
Ist $\lambda$ ein Eigenwert von A mit der algebraischen Vielfachheit m, dann besitzt die Gleichung: 
\begin{equation*}
(A-\lambda E)^m\nu = 0
\end{equation*}
eine Anzahl m linear unabhängige Lösungen $b_1$ bis $b_m$ und für k = 1...m sind die vektorwertigen Funktionen
\begin{equation*}
x_k(t) = e^{\lambda t}{b_k + \frac{t}{1!}(A-\lambda E) + ... + \frac{t^{m-1}}{(m-1)!}(A-\lambda E)^{m-1}b_k}
\end{equation*}
linear unabhängige Lösungen der Differentialgleichung. 
\subsection{Beispiel}
Folgende Matrix A ist gegeben: 
\begin{equation*}
	A =     
	\begin{vmatrix} 
	        4 & 6 & 6 \\ 
	        1 & 3 & 2 \\
	        -1 & -5 & -2    
	\end{vmatrix}
\end{equation*}
Die Matrix besitzt den einfachen Eigenwert $\lambda_1 = 1$ und den doppelten Eigenwert $\lambda_2=2$. Für den ersten Eigenwert erhält man den Eigenvektor $\nu_1 = [4 1 3]$ sowie die dazugehörige Lösung $x_1(t) = e^{\lambda_1 t}\nu_1$.\\
Zum zweiten Eigenwert gehört der Eigenvektor $\nu_2 = [3 1 -2]$, um zwei linear unabhängige Lösungen zu erhalten, halten wir uns an die Aussagen des obigen Theorems:\\ 
\textbf{Schritt 1:} \\
Man bestimme die beiden linear unabhängigen Lösungen $b_1$ und $b_1$ der Gleichung $(A-\lambda E)^2 \nu =0$.
Die Lösungen sind $b_1 =  [3 1 0]$ und $b_1 =  [0 0 1]$. \\
\textbf{Schritt 2:} \\
Mit Hilfe der beiden Vektoren bestimmen wir die beiden Lösungen: \\
$x_2(t) = e^{\lambda_2 t} \cdot {b_1 + \frac{t}{1!}(A-\lambda_2 E)b_1}$\\
$x_3(t) = e^{\lambda_2 t} \cdot {b_2 + \frac{t}{1!}(A-\lambda_2 E)b_2}$\\
\textbf{Schritt 3:} \\
Die Fundamentalmatrix X(t) aus den drei berechneten Lösungen zusammensetzen:\\
\begin{equation*}
	X(t) =     
	\begin{vmatrix} 
	        x_{11} & x_{21} & x_{31}\\ 
	        x_{12} & x_{22} & x_{32}\\
	        x_{13} & x_{23} & x_{33}    
	\end{vmatrix}
\end{equation*}
\subsection{Matrixexponential}
Gegeben ist die Differentialgleichung:
\begin{equation*}
\frac{d}{dt}x(t) = A(t)x(t)
\end{equation*}
mit Anfangbedingung $x(0) = x_0$ und der konstanten nxn Matrix A. Mit Hilfe des Flusses $\Phi(t)$ kann die Lösung dieses Problems dargestelltw erden als: 
\begin{equation*}
x(t) = \Phi(t)x_0
\end{equation*}
Bei einer skalaren Differentialgleichung $\frac{d}{dt}x(t) = ax(t)$ ist die Lösung bei gegebener Anfangsbedingung: $x(t) = e^{at}x_0$. Wenn wir diese Lösung nun mit $\Phi(t)$ vergleichen, stellt sich heraus, dass der Fluss eines nxn Systems ebenfalls eine Exponentialdarstellung besitzt. 
Ist A eine konstante nxn Matrix, dann besittz der Fluss $\Phi(t)$, der vom Vektorfeld erzeugt wird, folgende Darstellung als Matrixexponential: 
\begin{equation*}
\Phi(t) = e^{At} = X(t)X(0)^{-1}
\end{equation*}
Der Fluss wie auch das Matrixexponential können also mit Hilfe der Fundamentalmatrix X(t) und der entsprechenden Anfangsbedingung berechnet werden.
Damit hat die Lösung x(t) des Anfangswertproblems $x(0) = x_0$ folgende Form: 
\begin{equation*}
x(t) = e^{At}x_0
\end{equation*}
ToDo:\\
-Wronzki-Determinante / Fundamentalmatrix 
\subsection{Stabilit"at linearer Systeme}
Die Stabilit"at linearer Systeme $\diffp{}{t}x(t) = A \cdot x(t)$ mit regul"arer 2x2-Matix $A$\\

\begin{tabular}{|l|l|l|}
	\hline
	\textbf{Eigenwerte}                                  & \textbf{Klassifizierung}                & \textbf{Stabilit"at} \\ \hline
	$\lambda_1 > \lambda_2 > 0$                          & Knoten                                  & instabil             \\ \hline
	$\lambda_1 < \lambda_2 < 0$                          & Knoten                                  & asymptotisch stabil  \\ \hline
	$\lambda_2 <  0 < \lambda_1 $                        & Sattelpunkt                             & instabil             \\ \hline
	$\lambda_1 = \lambda_2  > 0$                         & eigentlicher oder uneigentlicher Knoten & instabil             \\ \hline
	$\lambda_1 = \lambda_2 < 0$                          & eigentlicher oder uneigentlicher Knoten & asymptotisch stabil  \\ \hline
	$\lambda_1 , \lambda_2 = \mu \pm i \cdot \nu$ mit    & Spiralpunkt                             &  \\
	$\mu > 0$                                            &                                         & instabil             \\
	$\mu < 0$                                            &                                         & asymptotisch stabil  \\ \hline
	$\lambda_1 = i\cdot \nu ,  \lambda_2 = -i \cdot \nu$ & Zentrum                                 & stabil               \\ \hline
\end{tabular} 

\subsection{Stabilit"at linearer Systeme mit reellen Koeffizienten}
\begin{subequations}
	\begin{equation*}
		\lambda^2 + p \lambda + q = 0
	\end{equation*}
	\begin{equation*}
		\lambda_{1,2} = \frac{1}{2} \cdot (p \pm \sqrt{\Delta}) \quad \text{mit} \quad \Delta = p^2 -4q
	\end{equation*}
\end{subequations}\\ \\

\begin{minipage}[h]{0.35\textwidth} 
	\includegraphics[width=0.9\textwidth]{images/StabilitaetsDiagramm.png}
\end{minipage}
\begin{minipage}[h]{0.35\textwidth}
	F"ur $\Delta > 0$ gilt:\\
	\begin{tabular}{|l|l|l|}
		\hline
		\textbf{Parameter $p$ und $q$} & \textbf{Eigenwerte}               & \textbf{Stabilit"at} \\ \hline
		$p,q > 0$                      & $\lambda_1 >0$ und $\lambda_2 >0$ & instabiler Knoten                        \\ \hline
		$p>0 ,  q<0$                   & $\lambda_1 >0$ und $\lambda_2 <0$ & Sattelpunkt                              \\ \hline
		$p< 0, q>0$                    & $\lambda_1 <0$ und $\lambda_2 >0$ & stabiler Knoten                          \\ \hline
		$p<0, q<0$                     & $\lambda_1 <0$ und $\lambda_2 >0$ & Sattelpunkt                              \\ \hline
	\end{tabular} \\ \\
	
	F"ur $\Delta = 0$ gilt:\\
	\begin{tabular}{|l|l|l|}
		\hline
		\textbf{Parameter $p$} & \textbf{Eigenwerte}        & \textbf{Stabilit"at}                         \\ \hline
		$p> 0$                         & $\lambda_1 = \lambda_2 >0$ & instabiler eigen-/ uneigentlicher Knoten \\ \hline
		$p< 0$                         & $\lambda_1 = \lambda_2 <0$ & stabiler eigen- / uneigentlicher Knoten   \\ \hline
	\end{tabular}\\ \\
	
	F"ur $\Delta < 0$ gilt:\\
		\begin{tabular}{|l|l|l|}
			\hline
			\textbf{Parameter $p$} & \textbf{Eigenwerte}                           & \textbf{Stabilit"at} \\ \hline
			$p> 0$                         & $\lambda_1 , \lambda_2 = \mu \pm i \cdot \nu$ & instabilen Strudel   \\ \hline
			$p< 0$                         & $\lambda_1 , \lambda_2 = \mu \pm i \cdot \nu$ & stabilen Strudel     \\ \hline
			$p= 0$                         & $\lambda_1 = \lambda_2 <0$                    & Zentrum              \\ \hline
		\end{tabular}  
\end{minipage}



\subsection{Entkoppeln}
\begin{minipage}[h]{0.35\textwidth}
Doppelter reeller Eigenwert:\\ Entkoppeln
\end{minipage}
\begin{minipage}[h]{0.5\textwidth}
	\includegraphics[width=1.0\textwidth]{images/Entkoppeln.png}
\end{minipage}

